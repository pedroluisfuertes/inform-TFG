\chapter*{}
%\thispagestyle{empty}
%\cleardoublepage

%\thispagestyle{empty}

%\input{portada/portada_2}



\cleardoublepage
\thispagestyle{empty}

\begin{center}
{\large\bfseries \myTitle: \mySubTitle}\\
\end{center}
\begin{center}
\myName\\
\end{center}

%\vspace{0.7cm}
\noindent{\textbf{Palabras clave}: Correo electrónico, ciberseguridad, phishing, malware, virus}\bigskip

\vspace{0.7cm}
\noindent{\textbf{Resumen}}\bigskip

El objetivo de este proyecto es intentar dar una ayuda extra en la identificación de correos maliciosos mediante la extracción y relación de características comunes. Por otro lado, se ofrecerá un servicio público y funcional para que los usuarios puedan aprovechar toda la información recopilada y analizar sus propios correos. 

Este proyecto surge de la falta de herramientas, tanto para usuarios técnicos como domésticos, a la hora de analizar un correo sospechoso una vez que llega a la bandeja de entrada. 

A lo largo del documento se hablará de los distintos tipos de correos maliciosos, de algunas técnicas usadas por los atacantes para engañar o manipular a sus víctimas, de los patrones que se van a extraer y cómo, de las relaciones que se van a hacer, …

En la parte de diseño se analizarán y compararán tanto lenguajes de programación como bases de datos, teniendo la idea en mente de migrar todo el servicio a la nube para tener un SaaS, siendo especialmente relevante el servicio de Azure Functions para ejecutar el código. 
En la parte funcional, el servicio debe permitir tanto analizar como buscar resultados, así como mostrar información relevante derivada del análisis.

También se indicarán problemas encontrados, soluciones aplicadas y posibles mejoras.

Finalmente, y como objetivo último se intentarán obtener ingresos económicos por parte del servicio de cara a crear una posible startup o venta del servicio.

\cleardoublepage


\thispagestyle{empty}


\begin{center}
{\large\bfseries \myTitle: \mySubTitle}\\
\end{center}
\begin{center}
\myName\\
\end{center}

%\vspace{0.7cm}
\noindent{\textbf{Keywords}: e-mail, cybersecurity, phishing, malware, virus}\bigskip

\vspace{0.7cm}
\noindent{\textbf{Abstract}}\bigskip

Write here the abstract in English.

\chapter*{}
\thispagestyle{empty}

%\noindent\rule[-1ex]{\textwidth}{2pt}\\[4.5ex]
\noindent\rule[-1ex]{\textwidth}{2pt}\bigskip\bigskip\bigskip

Yo, \textbf\myName, alumno de la titulación \myDegree de la \textbf{Escuela Técnica Superior
de Ingenierías Informática y de Telecomunicación de la Universidad de Granada}, con DNI XXXXXXXXX, autorizo la
ubicación de la siguiente copia de mi Trabajo Fin de Grado en la biblioteca del centro para que pueda ser
consultada por las personas que lo deseen.

\vspace{6cm}

\noindent Fdo: \myName

\vspace{2cm}

\begin{flushright}
Granada a X de mes de 201 .
\end{flushright}


\chapter*{}
\thispagestyle{empty}

\noindent\rule[-1ex]{\textwidth}{2pt}\bigskip\bigskip\bigskip

D. \textbf\myProf, Profesor del Área de XXXX del Departamento \myDepartment de la Universidad de Granada.

\vspace{0.5cm}

D. \textbf\myOtherProf, Profesor del Área de XXXX del Departamento \myDepartment de la Universidad de Granada.


\vspace{0.5cm}

\textbf{Informan:}

\vspace{0.5cm}

Que el presente trabajo, titulado \textit{\textbf{\myTitle, \mySubTitle}},
ha sido realizado bajo su supervisión por \textbf{\myName}, y autorizamos la defensa de dicho trabajo ante el tribunal
que corresponda.

\vspace{0.5cm}

Y para que conste, expiden y firman el presente informe en Granada a X de mes de 201 .

\vspace{1cm}

\textbf{Los directores:}

\vspace{5cm}

\noindent \textbf{\myProf \ \ \ \ \ \myOtherProf}

\chapter*{Agradecimientos}
\thispagestyle{empty}

       \vspace{1cm}


A mi familia, en especial a mi padre, a mi madre, a mi hermano y a mi hermana. 

A mis amigos.

A mis profesores.

