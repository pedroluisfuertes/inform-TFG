\chapter{Introducción}
El correo electrónico, desde que se creó, ha sido una fuente incesante de amenazas a los usuarios y, sin embargo, es el estándar de facto de la comunicación entre usuarios y empresas. Esto es debido a que cuando se crearon las primeras versiones de los protocolos SMPT y POP, en 1982 \citep{rfc_smtp} y 1984 \citep{rfc_pop} respectivamente, que se usan ampliamente en el envío y recepción de correos electrónicos, en internet apenas había una decena de miles de nodos conectados y la mayoría eran universidades y centros de investigación, por lo que la seguridad en aquel entonces no era un problema como lo es actualmente. 
Esa falta de seguridad en los primeros protocolos deriva en que en la actualidad el correo electrónico sea uno de los mayores agujeros de seguridad de las empresas y usuarios. 
Sumada a esta debilidad de la seguridad está la complejidad de los ataques, que son cada vez más complicados de detectar tanto por herramientas automatizadas como por los propios usuarios. 
\section{Tipos de amenazas mediante correo electrónico}
\subsection{SPAM}